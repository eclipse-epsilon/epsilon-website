With respect to the current situation discussed above, the context of the research hypothesis is as follows:

A Model Driven Engineering process can involve many different model management tasks such as validation, transformation (including model-to-model and model-to-text), in-place model transformation, comparison and merging. Currently, there are a number of independently developed task-specific languages that support some of these tasks, particularly model validation, model-to-model and model-to-text transformation. By contrast, no task-specific languages have been proposed for tasks such as model comparison, merging, inter-model validation and in-place transformation. Also, the fact that existing task-specific language have been predominantly developed independently of each other leads to consistency, reuse and interoperability problems.\\

In this context, the hypothesis of this thesis is stated as follows:\\

\textit{Despite their individual requirements and characteristics, a wide range of task-specific modelling languages share a significant number of common features, and therefore, instead of developing each language separately, it is beneficial in terms of reuse, uniformity and interoperability to develop them atop a platform that provides a reusable set of commonly required features.}\\
 
The objectives of the thesis are:

\begin{enumerate}
	\item To develop a platform atop which uniform, interoperable and reusable languages can be developed.
	\item To use the platform to develop task-specific languages that address the, now unsupported, tasks of inter-model consistency checking, model comparison, model merging and in-place model transformation.
	\item To use the platform to develop uniform task-specific languages for tasks that are already supported by existing languages (e.g. model-to-model transformation, model validation).
	\item To develop an orchestration and coordination framework that enables composition of individual model management tasks implemented using languages of the platform into coherent workflows. 
\end{enumerate}
